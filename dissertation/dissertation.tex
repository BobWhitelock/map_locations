\documentclass[12pt, a4paper]{report}
\usepackage[round]{natbib}

\title{Mapping Spatial References in Text using Google Maps}
\date{September 10, 2014}
\author{Bob Whitelock\\ MSc Software Systems and Internet Technology\\ Department of Computer Science\\ The University Of Sheffield}


\begin{document}

\maketitle

\begin{abstract}

\end{abstract}

\tableofcontents


\chapter{Introduction}
% Briefly set scene of background to project

% Outline aims, objectives of project

% Summarize remaining chapters


\chapter{Literature Review and Previous Work}
% Overview of chapter contents and reasons structured this way

% Review previous work and relationship to own
% Identify general trends and positions in area
% Overview of other available systems

% Background to georeferencing systems
\section{Georeferencing}
% Overview of what is involved in a georeferencing system

\cite{hill2006}

\section{Named Entity Recognition}
% What is involved, state of the art, STanford CoreNLP in particular

\section{Georeferencing Systems}
% Discuss various georeferencing systems and algorithms involved


\chapter{System Overview}
% Outline chapter: because more experimental project merge requirements and overview of design into this section, as requirements quite flexible and process more iterative than fixed requirements-design-implementation

\section{Requirements}
% Overview of requirements and aims of system in light of literature review

\section{Choice of Programming Language}
% Brief reasoning for choosing Python as main programming language for project

\section{System Architecture}
% Overview of architecture of system as series of components with overall pipeline through components, reasons and purpose for design

\section{System Evaluation Procedure}
% Description of procedure to be used for evaluating the effectiveness of the overall program and disambiguation strategies


\chapter{System Design and Implementation}
%Outline chapter: describe in detail process of medium to low-level design and implementation of system

\section{Content Acquisition}
% Means of text entry to system and reasons for choice. Process and problems in extracting text from arbitrary website

\section{Spatial Reference Recognition}
% Reasons for use of Stanford CoreNLP

% Problems with interfacing with CoreNLP, problems with using various wrappers and eventual solution

% Include brief description of how Stanford CoreNLP NER works/other techniques?


\section{Spatial Reference Identification}
% Describe process needed to correctly identify a spatial reference

\subsection{Choice of Gazetteer}
% Factors involved in choosing a gazetteer and reasons for choosing

% Choice of interface with gazetteer – REST API vs. database

% Creating and indexing database - include footnote to script used

\subsection{Disambiguation}
% What it means to disambiguate a locational reference


\section{Display of Results}
% Description of KML and reason for generating

% Generating KML, HTML, and viewing in web browser (mention use of third-party library as can’t reference local KML document with Google maps)


\chapter{Evaluation}
% Outline chapter: creation of evaluation setup and use to evaluate system

\section{Creation of Evaluation Environment}
% Process of adapting SpatialML corpus to be used for evaluation, and creating evaluation script
% Possible problems with evaluation process

\section{Evaluation of Disambiguation Algorithms}
% Description of each disambiguation algorithm, figures and graphs of results of evaluation, comparison between and reasons for difference

% Possible other algorithms, factors which could be used

\section{Evaluation of Overall System}
% How effective overall system is

% Possible improvements to system

\chapter{Conclusions}
% To include: lack of SpatialML system documentation, gazetteer, corpus hampers future work

\section{Conclusion}

\section{Future Work}


\bibliographystyle{plainnat}
\bibliography{dissertation.bib}


\end{document}


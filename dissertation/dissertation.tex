\documentclass[12pt, a4paper]{report}
\usepackage[round]{natbib}
\usepackage{glossaries}
\usepackage{graphicx} 

\title{Mapping Spatial References in Text using Google Maps}
\date{September 10, 2014}
\author{Bob Whitelock\\ MSc Software Systems and Internet Technology\\ Department of Computer Science\\ The University Of Sheffield}

%\makeglossaries
%\newglossaryentry{georeferencing}
%{
%	name=georeferencing
%	description={todo}	
%}


\begin{document}

\maketitle

\begin{abstract}

\end{abstract}

\tableofcontents

\glsaddall
\printglossaries

\chapter{Introduction}
% Briefly set scene of background to project

% Outline aims, objectives of project

% Summarize remaining chapters


\chapter{Literature Review and Previous Work}
% Overview of chapter contents and reasons structured this way




% Review previous work and relationship to own
% Identify gesec_neral trends and positions in area
% Overview of other available systems

% Background to georeferencing systems
\section{Overview of Georeferencing and Geoparsing}
\label{sec_overview_georeferencing}
% Overview of what is involved in a georeferencing system

% TODO redo this section

The aim of this project is to produce and evaluate a prototype geoparsing system; a system for the recognition and identification of spatial references in text. The system will also include means of inputting text to be parsed and of viewing the output of the system on a map. In this section we give an overview of geoparsing and the procedure involved, as well as outlining the benefits of both geoparsing and the wider field of georeferencing, the association of geographic information with text. 

*** FIX
\citet{hill2006} describes the importance of georeferencing of information as a means for assisting its understanding and interpretation. Increased georeferencing of information gives many benefits, as the understanding of information in terms of its spatial context  

Firstly this is due to the ubiquitous nature of spatial references in text, as well as the many benefits of increased access to this information  due to both the  Given the ubiquity of references to place in text,  

This importance stems from two directions. Firstly, it is worth noting that all events occur in space and time, and so improved access to spatial (and temporal) information can aid in the visualization and understanding of such events. In addition, it is also clear that "[g]eoreferenced information is everywhere" \citep{hill2006}, for instance: "[i]t has been estimated that at least 70 percent of our text documents contain placename references" (MetaCarta Inc., 2005a, cited in \citet[p.~5]{hill2006}). Together, these points indicate that greater georeferencing of information can lead to improved understanding of information in a wide variety of areas...
***

Geoparsing in particular is a subject that both improved techniques and wider adoption would be of great use in many areas of information retrieval and and analysis, especially given the ever-increasing number of digital documents available on the internet. The development of improved geoparsing software and the wider use of this could have applications in many areas, both in simplifying current tasks and enabling new actions to be performed that would be infeasible or impossible otherwise. For example, a geoparsing desktop application,  browser plugin, or website could allow users to easily visualise the geospatial events in any document they are reading; similarly such software could allow content creators to easily create a map of events in some content to accompany it, automating a task which would previously only be done manually if at all. In addition the use of geoparsing software by either content creators or search providers could allow improve the searching of content by geographic location, for instance by allowing the returning of relevant documents that contain references to locations in some area but do not reference the area itself. Furthermore, geoparsing of large bodies of text could enable analysis that would be impractical otherwise, such as analysing the spatial references in a large number of historical records.

\citet{hill2006} gives a detailed breakdown of the general steps involved in the geoparsing of a document, and these are as follows:

\begin{enumerate}
	\item {Digital text to be parsed is input to the system, and natural language techniques are used to identify features useful to the identification of spatial references, such as location names (by named-entity recognition), feature types (such as words like 'city' or 'mountain'), and other features of the text which may be useful (such as words indicating a relationship like 'in' or 'near' or directional or distance indicators like 'north of' or 'fifty miles east').}
	\item {The output of the initial step will be a number of spatial references, including but not necessarily limited to location names, and associated information from the text, and these should then be used to search for possible candidates for identification for each spatial reference, by lookup in a gazetteer (a geographic database containing location names as well as various associated information such as type of feature and geographic coordinates).}
	\item {If the gazetteer lookup is successful for a particular spatial reference then there will one or more candidates in the gazetteer for the identification of this reference.}
	\item {If more than one candidate has been found, an attempt must be made to disambiguate the most likely candidate to be that which the spatial reference is actually referring to; various techniques can be used for this, and both the various clues from the rest of the document and any other information given in the gazetteer can be used as factors in the disambiguation. If no candidates are found in the gazetteer for a spatial reference this disambiguation clearly cannot be done, and this is indicative of either limitations in the gazetteer or errors in the previous step of recognition of spatial references.}
	\item {Once a spatial reference has been identified as some gazetteer entry, coordinates can then be assigned to it with some degree of confidence. In addition, if desired an overall set of coordinates can be assigned to the document to georeference it as a whole; this could be done by different techniques, such as by an average of the individual identified coordinates or by identifying the location the document is primarily discussing and using the coordinates associated with this.}

\end{enumerate}


\section{Named Entity Recognition}
\label{sec_ner}
% What is involved, state of the art, Stanford CoreNLP in particular

One of the central components of a complete geoparsing system is a mechanism for the identification of spatial references in the text to be parsed. The key part of this component will be a means of recognising location names, as these are the most prominent type of spatial reference in most text, and in many cases the only type it will be possible to identify definitively. This problem is a specific instance of the general problem of named-entity recognition (NER); in this section an overview of NER is given with a summary of some of the main techniques in the field.

The task of NER is the automatic identification of parts of text that are the name of some named-entity, such as of a person, a location, or an organization, as well as for the classification of identified named-entities into groups such as these. What precisely is defined as a named-entity is hard to define definitively, however for the most part the definition of \citet{conll2002} will suffice, that '[n]amed entities are phrases that contain the names of persons, organizations, locations, times and quantities'. NER broadly consists of two parts, firstly the recognition of a sequence of tokens as a named-entity, and secondly the classification of named entities into particular types. It is by no means a simple task for numerous reasons, to name just some there are many different situations in which a named-entity can appear, the exact same string can refer to one type of entity in some situations and another in a different situation (for example "England" could refer to the country England or the organization of the English football team, depending on context), and difficulties in determining what constitutes a single named-entity (for example 'The Royal Bank of Scotland', which would correctly be identified as an organization, could easily be interpreted as two separate entities, an organization and a location, separated by 'of', due to the emphasis usually placed upon capitalization in determining named entities).

NER has received a considerable amount of work over the past twenty years, with a variety of techniques developed, many with considerable success in the particular area targeted. All techniques use various features of both words and a document to try to determine both the extent of and classification of named-entities in the document, often by the development of rules for recognition and classification based on these features. Such features can be of a variety of types and can take different types of values, including a boolean value, any of a range of valid values, or any string; the common factor however is that they are unambiguously determinable by a computer for. As described by \citet{nadeau2009}, common features used for recognition and classification can include:

\begin{itemize}
	\item {Word-level features: features of individual words, including: case of word (upper, lower, starts with a capital, mixed); any and type of punctuation in the word; any and type of digits in the word; types of other characters in the word; morphology of the word (the prefix, suffix, whether singular or plural); part-of-speech (the type of word, whether a foreign word); the application of some function over the word (such as extracting the non-alphabetic characters of the word).}
	\item {List-lookup features: various external sources to be used to assist in the recognition and classification of named-entities, including: general sources, such as dictionaries (e.g. to identify common nouns), lists of stop words, lists of common capitalized nouns, and common abbreviations; lists of various entities, such as names or parts of person, organization, or location names; lists of cues for entities, such as words typically found in organization names, person titles or common prefixes, and common words found in location names.}
	\item {Document features: features of the overall context of a word, the whole document, or the corpus as a whole, including: features of other occurences of a word, such as their case, and properties of coreferences of a word; the local syntax of a word; any meta information of a document, such as the url or information from tables or figures in the document; the frequency of words and phrases occurring in the document, and how often certain words occur together.}
\end{itemize}

Approaches to NER then use these various features to try to identify and classify named-entities in particular documents, usually by the  development of both recognition and classification rules which, applied to a document, should identify and classify named-entities based on the features of the document. As summarized by \citet{nadeau2009}, early approaches involved the use of 'heuristics and handcrafted rules', specifically tailored by the creator to try to tackle typical situations, determined by inspection of text in the domain to be tackled. More recently, however, the most common approach is to use various supervised machine learning techniques to train either a rule-based system or a sequence labelling algorithm, using a large quantity of human-annotated text as the training data. Some techniques in this approach include Hidden Markov Models, Decision Trees, Maximum Entropy Models, Support Vector Machines and Conditional Random Fields; the details of the majority of these is beyond the scope of this section.

There are also other techniques which require less or no human-annotated text to perform; these are important areas of development due to the time and expense needed to annotate new datasets for training use in purely supervised machine learning approaches. One such approach is semi-supervised learning, the main technique of which is known as 'bootstrapping'. This involves the use of a smaller dataset than with a purely supervised learning approach as seed example data to start the learning process, and then searches for similar examples to those in this set to reinforce the learning; some such approaches have had success close to a purely supervised approach. Finally there are unsupervised approaches, the main technique of which is known as 'clustering', which require no training data. These techniques mostly involve the use of lexical patterns and statistics on an unannotated corpus to enable recognition and classification of named-entities.

Given the advanced state of the NER field coupled with the limited nature of this project, it was decided to use already available software to initially fulfil this part of the system. This  will enable concentrating on implementing more novel aspects of the system rather than expending the imited resources of the project on reimplementing what is already available. For this purpose a number of 'off-the-shelf' NER systems are freely available, including Stanford NER \citep{finkel2005} which... [reasons chosen]

Of course for a geoparsing system to be comprehensive in its coverage of spatial references the identification of just named locations is not enough, even if this could be done with perfect accuracy, as there are many other types of spatial references that may be made in written text. To name just some these include:

\begin{itemize}
	\item { References relative to some named location, for example "fifty miles east of Sheffield". }
	\item { References to some unnamed location, which could potentially be determined from the context, such as to "the capital of Spain".}
	\item { Coreferences to a previously mentioned named location or other spatial reference, for example [TODO]}
	\item { An absolute reference to some geographic location using some formal georeferencing notation, for instance by latitude and longitude in some format such as "53.3836° N, 1.4669° W" or "(53.3836, -1.4669)" (both coordinates for Sheffield, England). }
\end{itemize}

As can be seen there can be many kinds of spatial reference beyond just the names of named locations. However incorporating any of these into a geoparsing system would require some amount of additional work on top of just using a NER system. Again due to the limited resources available for the project it was decided initially to build a full geoparsing system using just an NER system for the recognition of spatial references, namely Stanford NER. Once this was completed improving the recognition component of the system could be done, and this would be one way to improve the overall coverage of the geoparser. As it turned out we decided to concentrate our time on improving the disambiguation aspect of the system while keeping Stanford NER as the recognition component of the system, and so this direction of the project was not taken; this would therefore be one good extension to the project.

\section{Geoparsing Systems}
% Discuss various georeferencing systems and algorithms involved

A number of previous geoparsing systems have been constructed in the past few years, including research prototypes not intended for widespread use and industrial projects both proprietary and open-source. In this section we give an overview of a number of these, outlining the approaches taken by each.

\subsection{Research Systems}

One recent research project that had a reasonable amount of success at building a geoparsing system was the MIPLACE system created by a group at the MITRE Corporation \citep{mani2010}, a project to develop an automatic tagger for the SpatialML annotation scheme \citep{spatialml2009}. SpatialML is a markup language specifically designed for the comprehensive annotation of all spatial references in text, as well as any textually indicated relationships between spatial references. It includes a 'PLACE' tag to be used for marking up references to a specific location, covering both named references ("Sheffield") and nominal references ("a city"); it is also worth noting in particular that this tag includes a 'gazref' attribute for the specific association with a spatial reference of a gazetteer entry from some gazetteer. The scheme also includes a 'SIGNAL' tag for marking up specific spatial indicators such as "in" or "near", and non-consuming 'LINK' and 'RLINK' tags for indicating given topological (containment, connection) or trajectory (distance, direction) relationships respectively between  spatial references.

The SpatialML automatic annotator's major components follow the general structure for a geoparser as set out in \ref{sec_overview_georeferencing}, consisting of an 'Entity Tagger' followed by a 'Disambiguator' for the recognition and disambiguation of spatial references. In addition it also includes a final 'Relations Tagger' for adding in 'RLINK' and 'LINK' tags to a tagged document, and the ability to take HTML documents as input as well as to transform the SpatialML output into KML (Keyhole Markup Language) for viewing in map viewing software such as Google Earth. Both the Entity Tagger and the Disambiguator use supervised machine learning techniques to be trained to fulfill their purposes. They use feature vectors composed of various features of the document and the gazetteer, such as is described in \ref{sec_ner}; the data used to train the components is a human-annotated SpatialML corpus also produced as part of the project. [TODO more detail]

The MIPLACE system was evaluated against a ASC and ProMED SpatialML annotated corpora, for both recognition and and disambiguation accuracy. Recognition of spatial references was evaluated by considering the F-measure of the extent of references tagged, a binary decision which is successful only if the exact same text is identified as a spatial reference as in the corpus. For this the system scored F-measures of 86.9 and 67.54 on the ASC and ProMED corpora respectively. A number of reasons are given [TODO more]

[Refactor not very good] Despite the successes of the MIPLACE project, a number of ways it could be approved as a geoparser can be observed. For example, the method used to recognise spatial references is developed especially for this system, while the field of NER is quite advanced with sophisticated systems developed using means beyond the scope of just the MIPLACE project. To some extent it would appear necessary to have a custom built component to the disambiguator to retain the same functionality as MIPLACE currently, due to the tagging of nominal spatial references such as "a city". However, given that sophisticated and usable specialized NER systems have been developed, it would seem that gains to the recognition component of the system could be made by incorporating a specialist NER system into the recognition component rather than reimplementing this as part of the geoparser. Also with the disambiguation component of the system there is not a great deal of detail given about the particular machine learning algorithm used, however it is reasonable to suppose that this could be improved upon, especially given the many approaches taken to the subject in recent years [insert citations].

However there are a number of barriers to the MIPLACE system being a base geoparser to be improved upon. Firstly, while the source code of the system is available \footnote{http://sourceforge.net/projects/spatialml/}, it is undocumented in many ways, for example no documentation of the overall architecture of the system given and there is a minimal amount of comments within the code itself. In addition the system was observed to require a non-trivial amount of setup in a new environment in order to use. Furthermore, the gazetteer used for the disambiguation component of the system is the IGDB \citep{igdb2005}, which is not publically available. These factors combined mean that a considerable amount of effort would need to be expended to adapt the system in its current state in order to further develop it as a geoparser. Added to this, the MIPLACE system is under copyright to the MITRE Corporation so it is unclear if it would even be possible to pursue further development of the system. All these reasons together imply that, while MIPLACE is reasonably successful as a geoparser, there is still a need for a robust, documented, open-source and extendable geoparser in order to...? TODO

Another project to build a complete geoparser was undertaken by \citet{tobin2010}... [TODO explain]

[TODO add about that other project]

\subsection{Commercial Systems}

A number of commercial geoparsers have also been developed. These include the GeoTag function of MetaCarta's GSRP \footnote{http://www.metacarta.com/products-platform-geotag.htm} and Yahoo BOSS PlaceSpotter \footnote{https://developer.yahoo.com/boss/geo/docs/key-concepts.html}. GeoTag is described as a 'production-level geographic entity resolving function that parses content, extracts geographic references, and resolves the geographic meaning intended by the author', while PlaceSpotter is intended for 'identifying places in unstructured and "atomic" content ... and returning geographic metadata for geographic indexing and markup'. Both of these are commercial web services, and so of course are also closed source and little explanation is given about the procedure involved in either service, nor is any details of the success rate of either system given. As closed-source, commercial systems 
 
%Add?: Of course as a prototype system developed by a corporation it is understandable that the MIPLACE system would have these

\subsection{CLAVIN: An Open-Source System}

[TODO CLAVIN]

\chapter{Project Overview}
% Outline chapter: because more experimental project merge requirements and overview of design into this section, as requirements quite flexible and process more iterative than fixed requirements-design-implementation

In this chapter we give a high-level overview of various major aspects of the geoparser project. This structure is followed rather than splitting the different sections of the project into a strictly waterfall-like series of steps due to the more experimental and iterative nature of this project. As such there were less core requirements for the project, and the direction taken with the project was more open to adaptation as the project developed. This chapter begins with outlining the central requirements for the project, along with various possible optional directions it could proceed in. Next we give a brief discussion of the choice of programming language to be used for the project, before moving on to give a high-level overview of the architecture of the system in light of both the necessary requirements and possible extensions. Finally we give an overview of the evaluation procedure to be used for evaluating the success of the project.

\section{Requirements}
% Overview of requirements and aims of system in light of literature review

In this section we outline the requirements for the geoparser to have. As outlined above this project is of a somewhat experimental nature and does not have many core requirements aside from the general construction of a geoparser. As such the requirements for the project are fairly fluid, and so the requirements outlined here consist of a number of categories. Firstly there are core requirements for the construction of the geoparser, without which the project could not be considered complete. In addition we also outline other requirements we have chosen for various reasons, in terms of the architecture and additional features to be included in the system.

% put about requirements discursive rather than quantitative

The core of any geoparsing system must follow a structure similar to that set out in \ref{sec_overview_georeferencing}, with reference to \citet{hill2006}. Following this, first of all some means of inputting text to be parsed by the system is needed. One simple method to enable this would be to allow the system to take text files as input; this would enable easy processing by the system of any text file, so would be useful as a means of general input of text to be processed to the system. While file input alone would be enough to enable users of the system to process any text, a common way for people to read documents where georeferencing may be useful is using the internet. As such it would be very advantageous for the geoparser as a useful system to be able to take as input the URL of some HTML document obtainable over the Web, and then have a preliminary step of extracting the main textual content from this page before processing this through the remainder of the system. Because of these reasons it would be very useful to have both of these options for input methods to the system, and so both will be included.

After text has been input to the geoparser the next stage in the process is the recognition of spatial references in the text. While this could be taken to different levels depending on the extent of spatial references it is desired to identify, the core component of this stage is the NER of location names. As explained above this is a field with extensive previous work, including the availability of open-source NER systems such as Stanford NER which we have decided to use. As such initially this component will consist of using Stanford NER along with an interface to the rest of the program. [TODO more about extensions here?]

Once spatial references have been identified in the input text, the next stage in the geoparsing process is the identification of candidate locations for these. In order to do this some knowledge base is needed as a population to select candidates from; this is the role of the gazetteer in the system. The selection of the gazetteer can have a significant effect on the effectiveness of the geoparser. Choosing a gazetteer with good scope of coverage can improve the number of candidates found for a particular spatial reference and so the likelihood that the location which is really being referred to is present in the list. In addition choosing a gazetteer which provides as much information as possible about the different location entities means that there is more information to work with during the disambiguation stage of the geoparsing process. More information on the choice of gazetteer will be given in \ref{subsec_gazetteer}. The disambiguation step is the next stage of the process... TODO

Finally at the end of the previous step the result will be a text document along with certain identified spatial references in the document, that ideally will also be associated with some physical location as identified from a gazetteer. It would be very useful to a user of the geoparser to view this final output on a map, and so this should be the final step of the process

TODO more

\section{System Architecture}
% Overview of architecture of system as series of components with overall pipeline through components, reasons and purpose for design

\subsection{Overall Architecture}

\includegraphics[scale=0.5]{geoparser_architecture}

\subsection{Choice of Programming Language}
% Brief reasoning for choosing Python as main programming language for project

The choice of programming language or languages to be used for implementing any particular system is an important decision to be made before the implementation of the system can be undertaken. There are many different programming languages in active use today, and a myriad of factors in deciding which to use for the implementation of any particular project. Each programming language has had design decisions made in its development to make it more suitable for tackling different problem domains, as well as having different features in terms of which paradigms it supports and the expressiveness of its syntax. In addition to the core language, the capabilities of both the standard and available third-party libraries for the language will greatly effect the suitability of a language for a project, as the lack of library support for some feature may mean it will need to be implemented as part of the project when it would not otherwise.

Often one of the key factors effecting whether a language will be used however is only tangentially related to the language's design, as the capabilities of the team to undertake the project must be taken into account. This project is no exception, and as such as the sole developer there are only two languages I would feel capable of implementing the system in, either Java or Python.

Both of these languages are well-established with both having sophisticated standard and third-party libraries, as well as both being extensively used in NLP. One reason to favour Java as the primary language is that all of the Stanford CoreNLP tools, including Stanford NER which is to be used in the spatial reference recognition component of the system, are programmed in Java. Therefore use of Java would enable the calling of this code directly from the system. However, the CoreNLP tools are also runnable from the command line, and so it would not be strictly necessary to use Java code to interface with them. On the other hand, there are many reasons to favour Python for a project such as this. Python has an expressive syntax with many features with no equivalent in Java, such as list comprehensions and first-class functions, meaning it is often easier to manage complex logic due to less code being needed to express it. In addition its dynamic nature make it ideal for a more experimental project such as this, which is also not intended to be of such a size that the increased type safety of Java would outweigh this. For these reasons Python was chosen as the major implementation language for the project, as its expressiveness will allow the project to progress quickly and with less boilerplate code. 

\subsection{Choice of Gazetteer}
\label{subsec_gazetteer}
% Factors involved in choosing a gazetteer and reasons for choosing

As explained above, the choice of gazetteer to be used for drawing the candidate locations for a spatial reference from is an important one, effecting the likelihood that the correct location for any given spatial reference will be present, as well as the amount of information available on locations and so available to be used for disambiguation. There are a number of key criteria that should be taken into account for the selection of a gazetteer. As described by \citet{leidner2004} these include:

% number of different gazetteers available

\begin{enumerate}
	\item { Gazetteer availability: free resources should be favoured in research as they allow the sharing of data. This is true also for this project as the use of a free gazetteer will allow the free use of the system; additionally this is necessary as there is no budget for the project. }
	\item { Gazetteer scope: gazetteers can range in scope from local through regional and national to global, and a scope of at least the area that is being investigated is needed in order to provide full coverage. The system to be built is intended as a worldwide geoparser, and so a gazetteer with global coverage is necessary for this. }
	\item { Gazetteer completeness: gazetteers range in their comprehensive coverage of location names; for this project in particular it is clearly superior to have as comprehensive coverage as possible to increase the likelihood that a corresponding entry is present for any recognised location. }
	\item { Gazetteer correctness: geographic and administrative names used change over time, as well as the possibility that data can be input incorrectly in the first place}
\end{enumerate}

% Choice of interface with gazetteer – REST API vs. database

% Creating and indexing database - include footnote to script used


\section{System Evaluation Procedure}
% Description of procedure to be used for evaluating the effectiveness of the overall program and disambiguation strategies


\chapter{System Design and Implementation}
%Outline chapter: describe in detail process of medium to low-level design and implementation of system

In this chapter we describe the low-level design choices and important implementation details of each stage of the system. As described in [TODO design reference], we try to compartmentalize each major area of the system into distinct components for better modularity. The main logic of the program is contained within the \verb#map_locations.py# file, which includes the \verb#map_locations# function which performs the entire pipeline of the system, making calls to the other modules as needed and printing the current status of the system. Running this module is also how the system is run, by reading command-line arguments given and then executing the \verb#map_locations# function in accordance with these, and it is this which we mean by the overall system. [TODO more on overall pipeline]

\section{Content Acquisition}
% Means of text entry to system and reasons for choice. Process and problems in extracting text from arbitrary website

Before text can be processed by the system it must be input in some way, which two alternate methods are available for. First of these is simply to give input text by providing a file which the text is to be read from; this is done using the \verb#file# command-line option. This option is useful both for processing a single file and for batch-processing a number of files, such as is done as part of the evaluation process; it is also useful when batch-processing files to use the \verb#nomap# option so that the map produced as the output is not shown when the process is finished.

In addition to file input it is also possible to give a URL which the input text is to be read from, using the \verb#url# option instead. This option is particularly useful for practical use of the program, as the web is frequently the source of new reading that users may come across, and the provision of a map to accompany this reading could be a great aid to their understanding.

However processing text from the HTML at a given input URL is somewhat less simple than simply reading all of the text from a text file, due to the structure of the HTML. When a user gives a particular URL that they want processed, it can reasonably be assumed that they want only the main content at that URL to be processed and not any other small extraneous sections of text. However for most web pages with more than very minimal HTML, and especially for websites where the geoparser would be most useful such as news sites, there is a considerable amount of other text on the web page for any particular article, for example links to other parts of the site, small parts of other articles, reader comments and adverts. As such, simply stripping all HTML from a web page before processing it would always leave a considerable amount of extra material in the resulting text, which could pollute the input with unwanted text and so effect the output. This is especially likely since many of this other text could also potentially contain spatial references.

An obvious solution to this would be to parse the HTML for a given page to extract only the main content. However this would need to be done separately for every web site before the system could parse it and would also be easily broken by even small changes to the site's structure. An alternative solution is to develop a general approach to extracting the main content from web pages, by following rules as to how pages are laid out generally rather than trying to parse each particular page separately. One way to do this is by using the Readability Parser REST API \footnote{https://www.readability.com/developers/api/parser}, a freely available API which works very well for extracting the main content from a wide variety of web pages, and so rather than attempt to reimplement this we use this API  to perform this task. The API is accessed in the \verb#readability_interface# module, using the Requests \footnote{http://docs.python-requests.org/en/latest/} Python library to make HTTP requests for an article's content.

\section{Spatial Reference Recognition}
% Reasons for use of Stanford CoreNLP

% Problems with interfacing with CoreNLP, problems with using various wrappers and eventual solution

% Include brief description of how Stanford CoreNLP sec_ner works/other techniques?


\section{Spatial Reference Identification}
% Describe process needed to correctly identify a spatial reference

% more about gazetteer here?

\subsection{Disambiguation}
% What it means to disambiguate a locational reference


\section{Display of Results}
% Description of KML and reason for gesec_nerating

% Generating KML, HTML, and viewing in web browser (mention use of third-party library as can’t reference local KML document with Google maps)


\chapter{Evaluation}
% Outline chapter: creation of evaluation setup and use to evaluate system

\section{Creation of Evaluation Environment}
% Process of adapting SpatialML corpus to be used for evaluation, and creating evaluation script
% Possible problems with evaluation process

\section{Evaluation of Disambiguation Algorithms}
% Description of each disambiguation algorithm, figures and graphs of results of evaluation, comparison between and reasons for difference

% Possible other algorithms, factors which could be used

\section{Evaluation of Overall System}
% How effective overall system is

% Possible improvements to system

\chapter{Conclusions}
% To include: lack of SpatialML system documentation, gazetteer, corpus hampers future work

\section{Conclusion}

\section{Future Work}


\bibliographystyle{plainnat}
\bibliography{dissertation.bib}


\end{document}

